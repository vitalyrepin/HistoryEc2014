\documentclass[a4paper]{article}
\usepackage[T1,T2A]{fontenc}
\usepackage[utf8]{inputenc}
\usepackage[english,russian]{babel}
\usepackage{cmap}
\usepackage{multicol}
\usepackage{hyperxmp}
\usepackage{hyperref}
\usepackage[landscape,left=0.1cm, right=.1cm, top=0.1cm, bottom=1cm,paperwidth=210mm,paperheight=297mm]{geometry}
\usepackage{stmaryrd}
\usepackage{verbatim}
\usepackage{tabularx}
\usepackage{amsmath}
\usepackage{amssymb}
\usepackage{bm}
\usepackage{fancyhdr}

\chead{}\rhead{}\lhead{}
\cfoot{\vspace*{-1cm}\small\ccbysa\ Vitaly Repin, 2014. This work is licensed under a \href{http://creativecommons.org/licenses/by-sa/3.0/}{Creative Commons Attribution-ShareAlike 3.0 License}. Created with \LaTeXe. \href{https://github.com/vitalyrepin/HistoryEc2014/tree/master/wk8Schemas}{Sources.}\\ По материалам лекций проф. Макашевой (ВШЭ): \href{https://www.coursera.org/course/historyofec}{История экономической мысли: <<Джон Мейнард Кейнс: революционер-реформатор>>.}}
\pagestyle{fancy}

\usepackage{tikz}
\usepackage{ccicons}
\usetikzlibrary{mindmap}
\usetikzlibrary{arrows}

\newcommand{\upperRomannumeral}[1]{\uppercase\expandafter{\romannumeral#1}}

\definecolor{historyfg}{HTML}{77797C}
\definecolor{historybg}{HTML}{EEEEEE}
\definecolor{scibg}{HTML}{EEEEEE}

\hypersetup{%
unicode=true,           % русские буквы в раздела PDF
pdftitle={%
История экономической мысли (лекции ВШЭ): Кейнс и кейнсианство.},
pdfauthor={Vitaly Repin},
pdfcopyright={This work is licensed under Creative Commons Attribution-ShareAlike 3.0 License},
pdfsubject={История экономической мысли: Д.М. Кейнс},
pdfkeywords={Кейнс, кейнсианство},
pdflicenseurl={http://creativecommons.org/licenses/by-sa/3.0/},
pdfcaptionwriter={Vitaly Repin},
pdfcontactcity={Espoo},
pdfcontactcountry={Finland},
pdfcontactemail={vitaly.repin@gmail.com},
pdflang={ru},
colorlinks=true,       	% false: ссылки в рамках; true: цветные ссылки
linkcolor=cyan,          % внутренние ссылки
citecolor=green,        % на библиографию
filecolor=magenta,      % на файлы
urlcolor=blue           % на URL
}

\begin{document}
\renewcommand{\headrulewidth}{0mm}
\pgfdeclarelayer{background}
\pgfsetlayers{background,main}

\begin{center}
\begin{tikzpicture}[mindmap,scale=1.2,
every node/.style={concept,execute at begin node=\hskip0pt},
concept45/.style={concept, sibling angle=45},
root concept/.append style={
concept color=black, fill=white, line width=1ex, text=black,minimum size=3cm,text width=3cm,font=\large\scshape},
text=historyfg,
level 1/.append style={sibling angle=120, level distance=3.2cm},
level 2/.append style={minimum size=3cm, text width=2.7cm,level distance=3.1cm, font=\small},
level 3/.append style={minimum size=1.3cm,level distance=3.2cm, text width=2cm, font=\small},
level 4/.append style={minimum size=1.1cm, text width=3cm},
concept color=historybg,grow cyclic,
remark/.style={
rectangle,
sloped,
minimum size=4mm,
very thin,
color=white,
text=black,
font=\itshape
}]
\node[root concept] {\href{http://ru.wikipedia.org/wiki/%D0%9A%D0%B5%D0%B9%D0%BD%D1%81%2C_%D0%94%D0%B6%D0%BE%D0%BD_%D0%9C%D0%B5%D0%B9%D0%BD%D0%B0%D1%80%D0%B4}{Д.М. Кейнс (1883--1946)}}
   child[concept, sibling angle=120] { node{Философия}
  	child { node[text=brown] (uncert) {Интерес к проблеме неопределенности}
		child[level distance=3.4cm] { node {(Вероятностная) логика нестрогого следования: <<Трактат о вероятности>> (1921)} }
		child[sibling angle=90, level distance=3.4cm] { node {Знание = логика + интуиция. Интуиция важна для агентов фин. рынка!}}
	}
	child { node {Отказ от идеи построения универсальной картины мира} }
    }
    child { node {Деньги и \%}
	child { node[text=brown] (pastcurr) {Деньги --- связь между настоящим и будущим}
		child[sibling angle=90] {node {\href{http://ru.wikipedia.org/wiki/\%D0\%A2\%D0\%B5\%D0\%BE\%D1\%80\%D0\%B8\%D1\%8F_\%D0\%B4\%D0\%B5\%D0\%BD\%D0\%B5\%D0\%B3\%23.D0.9A.D0.BE.D0.BB.D0.B8.D1.87.D0.B5.D1.81.D1.82.D0.B2.D0.B5.D0.BD.D0.BD.D0.B0.D1.8F_.D1.82.D0.B5.D0.BE.D1.80.D0.B8.D1.8F_.D0.B4.D0.B5.D0.BD.D0.B5.D0.B3_.28.D0.BA.D0.BE.D0.BD.D0.B5.D1.86_17_.D0.B2._-_.D0.BD.D0.B0.D1.87.D0.B0.D0.BB.D0.BE_18_.D0.B2._-....29}{Количественная теория денег} $C=S/V$ этого не видит}}
		child {node {Кейнс меняет взгляд на задачи ден. политики}}
	}
	child[text=brown] { node (percent) {\% --- выражение предпочтения между настоящим и будущим }}
        child { node {Деньги ведут себя иначе, чем др. товары (чем они дороже, тем выше спрос!) }}
    }
    child[concept] { node {Отказ от викторианской этики}
	child { node { Общее благо $\neq\Sigma$ благ индивидов}  }
	child { node (indiv) { Критика рационалистического индивидуализма}  }
	child { node (save) { Накопление не есть абсолютная добродетель }  }
	child[level distance=4cm] { node { Экономическую политику должны определять \textbf{\underline{не}} абсолютные принципы, а \textbf{целесообразность в сложившихся условиях} }  }
  	child[level distance=4cm] { node[text=brown] (faire) {Отказ от \href{http://ru.wikipedia.org/wiki/Laissez-faire}{принципа Laissez-faire}: \textcolor{brown}{\textbf{необходимо} $\uparrow$ роли государства для $\downarrow$ рисков и неопределенности}}}
  };
\begin{scope}[every annotation/.style={font=\normalsize, fill=white, text width=9cm}]
\node [annotation, right] at (indiv.east) {
Рациональные действия, направленный на $\uparrow$ вашего богатства могут привести ко всеобщей (в т.ч. вашей!) бедности! Пр.: никто не тратит деньги, ожидая $\downarrow$ цен.
};
\node [annotation, left] at (save.west) {
Раньше (Кейнс, 1926): <<богатый человек мог попасть в Царствие Небесное, если только он делал сбережения>>.};
\end{scope}
\begin{pgfonlayer}{background}
	\draw [concept connection] (uncert) edge (faire);
\end{pgfonlayer}
\node[remark, text width=7.4cm] at (-10, -1) {\parbox{7.3cm}{\emph{Кейнс, 1925:} {<<Я направляю все свои усилия и внимание на поиск новых путей и новых идей относительно перехода от экономической анархии индивидуалистического капитализма\dots к режиму, который имеет целью контроль и направление экономических сил \textbf{в интересах социальной справедливости и устойчивости}>>.}}};
\node[remark, text width=11cm] at (3, -9) {Лекция 8.5. Философия, этика и «Денежная» трилогия Кейнса};
\end{tikzpicture}
\end{center}
\newpage

\newgeometry{left=1cm, right=1cm, top=.8cm, bottom=1cm}

\setlength{\columnseprule}{.4pt}
\centerline{\large\it Лекция 8.6. «Общая теория занятости, процента и денег» (1936). Слайд \upperRomannumeral{1}.}
\smallskip

\begin{center}
\begin{tabular}{|p{.49\textwidth}|p{.49\textwidth}|}
\hline
\Large\sc Совокупный спрос & \Large\sc Деньги и процент\\
\hline
Сбережения $\neq$ спрос! (разрыв с классикой, споры с австрийцами)

\begin{description}
\item[\textbf{Инвестиционный спрос I}] {--- потенциальная потребность предпринимателей, выраженная в денежной форме, в приобретении товаров инвестиционного назначения с целью получения дохода (\textbf{Y}).}
\item[\textbf{Потребительский спрос C}] { ---  часть совокупного спроса в экономике, относящаяся к потребительским товарам.}
\end{description}

\noindent \textbf{Совокупный спрос} зависит от \textbf{I} и \textbf{C,} но \textbf{I} - основной компонент!
\smallskip

\noindent Компоненты инвестиционного спроса $I:$
\begin{itemize}
\item \textbf{Предельная эффективность капитала $\bm{\eta}$} --- наиболее высокая процентная ставка $\bm{i}$, при которой можно рассчитывать на окупаемость проекта
\item \textbf{Мультипликатор Кейнса $\bm{\mu}$} показывает, насколько изменение инвестиций способствует изменению дохода. $\mu = \text{ПСС}^{-1}.$ Высокая ПСС снижает $\mu$ и потому опасна!
\begin{itemize}
\item \textbf{Предельная склонность к сбережению ПСС (MPS)} --- часть полученной населением дополнительной денежной единицы, направляемая на дополнительное сбережение, $\text{ПСС} = 1-\text{ПСП}$
\item \textbf{Предельная склонность к потреблению ПСП (MPC)} --- часть полученной населением дополнительной денежной единицы, направляемая на дополнительное потребление, $\text{ПСП} = 1-\text{ПСС}$
\item \href{http://ru.wikipedia.org/wiki/\%D0\%9E\%D1\%81\%D0\%BD\%D0\%BE\%D0\%B2\%D0\%BD\%D0\%BE\%D0\%B9_\%D0\%BF\%D1\%81\%D0\%B8\%D1\%85\%D0\%BE\%D0\%BB\%D0\%BE\%D0\%B3\%D0\%B8\%D1\%87\%D0\%B5\%D1\%81\%D0\%BA\%D0\%B8\%D0\%B9_\%D0\%B7\%D0\%B0\%D0\%BA\%D0\%BE\%D0\%BD}{Основной психологический закон} --- процент дохода, направляемый на сбережения, растёт по мере роста доходов.
\end{itemize}
\item \textbf{Инвестиции и доход (Y)}
\end{itemize}

Экономикой движет \emph{прежде всего} $I.$
&
Деньги - не только средство обращения, но это ликвидный актив. Альтернатива вложению в длительные, рискованные активы.

\begin{itemize}
\item \textbf{\href{http://ru.wikipedia.org/wiki/\%D0\%9F\%D1\%80\%D0\%B5\%D0\%B4\%D0\%BF\%D0\%BE\%D1\%87\%D1\%82\%D0\%B5\%D0\%BD\%D0\%B8\%D0\%B5_\%D0\%BB\%D0\%B8\%D0\%BA\%D0\%B2\%D0\%B8\%D0\%B4\%D0\%BD\%D0\%BE\%D1\%81\%D1\%82\%D0\%B8}{Предпочтение ликвидности} $\bm{\lambda}$} --- показатель спроса на денежные средства, трактуемые как ликвидность.
\item \textbf{\textit{M}} - количество денег в обращении.
\item \textbf{Процент \textit{i}} --- плата за расставание с ликвидностью. $i \varpropto \lambda,$ так как чем больше нежелание расставаться с денежной формой богатства, тем выше процент как вознаграждение за отказ от ликвидности; $i\varpropto M^{-1},$ так как норма процента $\downarrow$ по мере роста количества
денег и $\uparrow$ при увеличении спроса на деньги по сравнению с их предложением.
\item Курсы ценных бумаг влияют на \%, под который реальный сектор получает кредиты (влияние финансового рынка  (<<казино>>) на реальный). В частности, \emph{могут помешать понизить кредитную ставку для стимулирования инвестиций.}
\item <<Простая>> модель: $M - (\lambda) \rightarrow i - (\eta) \rightarrow I - (\mu) \rightarrow Y,$ где $\mu, \lambda$ и $\eta$ --- экзогенные (внешние) параметры.
\begin{itemize}
\item При заданной $\mu\, \text{(ПСП),}$ уровень занятости и доход $Y$ определяются $I$
\item При заданной $\eta,$ инвестиционный спрос $I$ определяется процентом $i$
\item При заданном $\lambda,$ процент $i$ определяется денежной массой $M$
\end{itemize}
\item \textbf{\href{http://ru.wikipedia.org/wiki/\%D0\%9B\%D0\%BE\%D0\%B2\%D1\%83\%D1\%88\%D0\%BA\%D0\%B0_\%D0\%BB\%D0\%B8\%D0\%BA\%D0\%B2\%D0\%B8\%D0\%B4\%D0\%BD\%D0\%BE\%D1\%81\%D1\%82\%D0\%B8}{Ловушка ликвидности}} --- ситуация, когда монетарные власти не имеют инструментов для стимулирования экономики, ни через $\downarrow i,$  ни через $\uparrow M.$ Обычно возникает, когда ожидания негативных событий (напр., дефляции) ведут к $\uparrow\lambda.$
\item \textbf{Кризис} --- следствие неоправдавшихся ожиданий инвесторов, а вовсе не переинвестирование (позиция классиков). Избыток инвестиций не м.б. абсолютным, но инвестиции м.б. избыточными \emph{по сравнению} с ожиданиями. Доп. фактор: лавинообразное распространение пессимизма.
\end{itemize}\\
\hline
\end{tabular}
\end{center}

\newpage
\centerline{\large\it Лекция 8.6. «Общая теория занятости, процента и денег» (1936). Слайд \upperRomannumeral{2}.}
\smallskip
\begin{center}
\begin{tabular}{|p{.49\textwidth}|p{.49\textwidth}|}
\hline
\multicolumn{2}{|c|}{\Large{\textsf{\mathstrut Переворот представлений о рынке труда}}}\\
\hline
\multicolumn{1}{|l|}{\Large\sc\mathstrut Классики} & \multicolumn{1}{l|}{\Large\sc\mathstrut Кейнс}\\
\hline
\begin{itemize}
\item Спрос и предложение обеспечивают равновесие на рынке труда;
\item Безработица в классической модели имеет добровольный характер, поскольку ее причиной выступает отказ рабочего работать за данную ставку номинальной з/п.
\end{itemize}
&
\begin{itemize}
\item Предложение труда определяется \emph{номинальной} з/п, а спрос на труд --- \emph{реальной;}
\item \emph{Парадокс макро и микро:} Если $\downarrow$ номинальной з/п приведет к $\uparrow$ реальной з/п (напр., из-за снижения цен на товары вследствие $\downarrow$ издержек на з/п), то фирмы начнут увольнять людей и безработица $\uparrow;$
\item \framebox{Устранение безработицы ценовыми методами невозможно!}
\end{itemize}\\
\hline
\end{tabular}
\end{center}

\medskip

{\large\sc Выводы}

\begin{itemize}
\item \emph{Парадокс макро и микро:} \framebox{Высокая ПСС снижает $\mu$ и потому опасна (способствует спаду)!}
\item Не бережливость, а \framebox{предприимчивость --- движущая сила процесса накопления!}
\item Главный практический вывод: \framebox{опасно (для экономики и общества) оставлять в частных руках регулирование объема инвестиций.}
\item \framebox{Цель экономической политики: увеличение производства и занятости.}\\[.1cm] Способы достижения:
\begin{itemize}
\item Прежде всего стимулирование инвестиционного спроса $I$ (фискальная политика для $\uparrow$ ПСП,
$\downarrow i,$ но см. также \href{http://ru.wikipedia.org/wiki/\%D0\%9B\%D0\%BE\%D0\%B2\%D1\%83\%D1\%88\%D0\%BA\%D0\%B0_\%D0\%BB\%D0\%B8\%D0\%BA\%D0\%B2\%D0\%B8\%D0\%B4\%D0\%BD\%D0\%BE\%D1\%81\%D1\%82\%D0\%B8}{ловушку ликвидности}).
\item Если перераспределять доходы в пользу бедных или государства, то они будут не сберегать, \emph{а тратить} $\Rightarrow$ \framebox{фискальная политика может способствовать $\uparrow \mu.$}
\\ А это означает, что социальные и экономические цели не конфликтуют (в отличие от классиков)!
\item \textsl{\small/не надо воспринимать как гимн в пользу инфляции/} $\uparrow M$ для обслуживания возросшего денежного дохода (в случае успеха других мер). Иначе недостаточная денежная масса $M$ будет способствовать $\uparrow i.$
\item Институциональные меры, направленные на $\downarrow$ неопределенности (напр., страхование депозитов, регулирование фин. рынка), создание благоприятного климата (напр., большие гос. проекты
имеют не только прямой, но и косвенный эффект --- стимулирование <<инвестиционного бума>>).
\end{itemize}
\end{itemize}



\end{document}
