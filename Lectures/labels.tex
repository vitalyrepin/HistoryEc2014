\documentclass[a4paper]{article}
\usepackage[utf8]{inputenc}
\usepackage[english,russian]{babel}
\usepackage{a4wide}
\setlength{\topmargin}{0pt}
\setlength{\headheight}{0pt}
\addtolength{\textheight}{4cm}
\voffset=-2cm
\pagestyle{empty}

\begin{document}
\Large

\begin{center}
\hrule
\medskip

\textbf{\textsc{История экономической мысли}}
\smallskip

{\large\textrm{Ананьин О.И. (Высшая Школа Экономики)}}

\smallskip

\textsl{Весна 2014\,г. Недели 1--3.5}
\end{center}

\hrule
\newpage


\begin{center}
\hrule
\medskip

\textbf{\textsc{История экономической мысли}}
\smallskip

{\large\textrm{Ананьин О.И. (Высшая Школа Экономики)}}

\smallskip

\textsl{Весна 2014\,г. Недели 1--3.5}
\end{center}

\large

\centerline{\parbox{14cm}{\begin{description}
\itemsep-.2cm
\item[Week 1] \hspace*{1cm}
\begin{itemize}
\item[1.1] Экономика как наука: муки рождения (16:17)
\item[1.2]  Осмысление экономической политики: меркантилизм (16:14)
\item[1.3]  Меркантилизм: анахронизм или зерно истины? (12:31)
\item[1.4]  Эмпирическая или инженерная наука? (17:34)
\item[1.5]  Система Кантильона против <<системы>> Ло (8:51)
\item[1.6]  Детерминизм и неопределенность (14:03)
\item[1.7]  Физиократы: идея естественного закона (8:17)
\item[1.8]  Экономическая таблица Ф.Кенэ (12:49)
\item[1.9]  Судьба Жака Тюрго, теоретика и министра (9:03)
\end{itemize}
\item[Week 2]\hspace*{1cm}
\begin{itemize}
\item[2.1] Адам Смит и экономическая классика (19:27)
\item[2.2]  Возможные пути развития экономической науки (8:42)
\item[2.3]  В контексте теории (11:26)
\item[2.4]  Схема производства богатства (9:22)
\item[2.5]  Как измерить рост продукта? (13:18)
\item[2.6]  Теория естественной цены, или стоимости (11:30)
\item[2.7]  Послесмитовская эпоха (15:29)
\item[2.8]  Теория земельной ренты: предыстория (24:47)
\item[2.9]  Давид Рикардо (19:59)
\end{itemize}
\item[Week 3]\hspace*{1cm}
\begin{itemize}
\item[3.1]  Феномен Карла Маркса (16:40)
\item[3.2]  На плечах гигантов (11:28)
\item[3.3]  Диалектика <<Капитала>>: стоимость (17:10)
\item[3.4]  Диалектика <<Капитала>>: всеобщая формула (15:06)
\item[3.5]  Диалектика <<Капитала>>: накопление (19:46)
\end{itemize}
\end{description}}}
\hrule

\end{document}
