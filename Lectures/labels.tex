\documentclass[a4paper]{article}
\usepackage[utf8]{inputenc}
\usepackage[english,russian]{babel}
\usepackage{a4wide}
\setlength{\topmargin}{0pt}
\setlength{\headheight}{0pt}
\addtolength{\textheight}{4cm}
\voffset=-2cm
\pagestyle{empty}

\begin{document}
\Large

\newcommand{\hist}{
\begin{center}
\hrule
\medskip

\textbf{\textsc{История экономической мысли}}
\smallskip

{\large\textrm{Ананьин О.И. (Высшая Школа Экономики)}}

\smallskip

\textsl{Весна 2014\,г. Недели 1--3.5}
\end{center}}

\hist

\hrule
\newpage

\hist

\large

\centerline{\parbox{14cm}{\begin{description}
\itemsep-.2cm
\item[Week 1] \hspace*{1cm}
\begin{itemize}
\item[1.1] Экономика как наука: муки рождения (16:17)
\item[1.2]  Осмысление экономической политики: меркантилизм (16:14)
\item[1.3]  Меркантилизм: анахронизм или зерно истины? (12:31)
\item[1.4]  Эмпирическая или инженерная наука? (17:34)
\item[1.5]  Система Кантильона против <<системы>> Ло (8:51)
\item[1.6]  Детерминизм и неопределенность (14:03)
\item[1.7]  Физиократы: идея естественного закона (8:17)
\item[1.8]  Экономическая таблица Ф.Кенэ (12:49)
\item[1.9]  Судьба Жака Тюрго, теоретика и министра (9:03)
\end{itemize}
\item[Week 2]\hspace*{1cm}
\begin{itemize}
\item[2.1] Адам Смит и экономическая классика (19:27)
\item[2.2]  Возможные пути развития экономической науки (8:42)
\item[2.3]  В контексте теории (11:26)
\item[2.4]  Схема производства богатства (9:22)
\item[2.5]  Как измерить рост продукта? (13:18)
\item[2.6]  Теория естественной цены, или стоимости (11:30)
\item[2.7]  Послесмитовская эпоха (15:29)
\item[2.8]  Теория земельной ренты: предыстория (24:47)
\item[2.9]  Давид Рикардо (19:59)
\end{itemize}
\item[Week 3]\hspace*{1cm}
\begin{itemize}
\item[3.1]  Феномен Карла Маркса (16:40)
\item[3.2]  На плечах гигантов (11:28)
\item[3.3]  Диалектика <<Капитала>>: стоимость (17:10)
\item[3.4]  Диалектика <<Капитала>>: всеобщая формула (15:06)
\item[3.5]  Диалектика <<Капитала>>: накопление (19:46)
\end{itemize}
\end{description}}}
\hrule
\newpage

\renewcommand{\hist}{
\begin{center}
\hrule
\medskip

\textbf{\textsc{История экономической мысли}}
\smallskip

{\large\textrm{Ананьин О.И., Автономов В.С., Макашева Н.А.(Высшая Школа Экономики)}}

\smallskip

\textsl{Весна 2014\,г. Недели 3.6--7.13 (кроме недели 4)}
\end{center}}

\hist

\hrule
\newpage

\hist

\centerline{\parbox{15cm}{\begin{description}
\itemsep-.2cm
\item[Week 3] \hspace*{1cm}
\begin{itemize}
\item[3.6]  <<Капитал>> как критика политэкономии (12:56)
\item[3.7] Фетишизм политэкономии (14:13)
\item[3.8] Прогнозы Маркса (11:58)
\item[3.9] Судьба теории Маркса (6:53)
\end{itemize}
\item[Week 5] \hspace*{1cm}
\begin{itemize}
\item[5.1] Маржинализм: пример революции в науке (12:55)
\item[5.2] Классики и маржиналисты: сопоставление подходов (8:52)
\item[5.3] Предшественники и последователи. причины и последствия (13:14)
\item[5.4] Австрийская школа: К. Менгер (17:08)
\item[5.5] Австрийская школа: обмен (13:13)
\item[5.6] Л. Вальрас и Лозаннская школа (22:57)
\item[5.7] А. Маршалл – завершитель революции и создатель неоклассики (30:34)
\item[5.8] Маржиналистская теория благосостояния (9:26)
\end{itemize}
\item[Week 6] \hspace*{1cm}
\begin{itemize}
\item[6.1] Два канона экономической науки (12:10)
\item[6.2] Т. Веблен: <<Теория праздного класса>> (27:11)
\item[6.3] Американский институционализм: Дж. Коммонс и Митчелл (13:57)
\item[6.4] Й. Шумпетер: жизнь и творчество (16:10)
\item[6.5] Й. Шумпетер: <<Теория экономического развития>> (27:09)
\item[6.6] Й. Шумпетер: <<Капитализм, социализм и демократия>> (19:22)
\end{itemize}
\item[Week 7] \hspace*{1cm}
\begin{itemize}
\item[7.1]  Методологический комментарий (15:12)
\item[7.2]  Хронологический комментарий (12:49)
\item[7.3]  Российская экономическая мысль в начале XX века (31:14)
\item[7.4]  Важнейшие политико-экономические вопросы (8:59)
\item[7.5]  Дискуссии о теории ценности (12:58)
\item[7.6]  М.И. Туган-Барановский: попытки синтеза (12:1)
\item[7.7]  В.К.Дмитриев: попытки синтеза (10:01)
\item[7.8]  П.Б.Струве: расставание с идеей синтеза (11:04)
\item[7.9]  А.Д.Билимович: дедуктивная теория цены (7:20)
\item[7.10]. Теория распределения и теория ценности: два подхода (7:05)
\item[7.11]  М.И. Туган-Барановский: Теория циклов и кризисов (14:54)
\item[7.12]  Экономическая наука в новых политических условиях (16:54)
\item[7.13]  Н.Д.Кондратьев: методология и большие циклы (14:28)
\end{itemize}
\end{description}}}
\hrule

\end{document}
